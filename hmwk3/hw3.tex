\documentclass[11pt]{article}

\usepackage[utf8]{inputenc}

\usepackage[
  margin=1.5cm,
  includefoot,
  footskip=30pt,
]{geometry}
\geometry{a4paper}

\usepackage{verbatim}
\usepackage{hyperref}
\usepackage{amsmath}
\usepackage{qtree}
\usepackage{semantic}
\usepackage{mathpartir}

\title{HW3 -- Operational Semantics}
\author{Robert David Hernandez, CS 476, Fall 2023}
\date{}

\begin{document}
\maketitle

\section{Instructions}
This is a written assignment. You may write your solutions by hand and scan/take a picture of the paper, or in a text editor (Notepad, Word, LaTeX, etc.) and submit a text file or PDF. If you need any help getting your solutions into a suitable format, just let the instructors know. As always, please don't hesitate to ask for help on Piazza (\url{https://piazza.com/class/ksknvqg6ogb2kc}).

\section{Operational Semantics of Expressions and Commands}
Here are the operational semantics rules for a simple imperative programming language, using the ``hybrid style'' of big steps for expressions and small steps for commands.

\begin{mathpar}
\inference[\textsc{num}]{(n\text{ is a number literal})}{(n, \rho) \Downarrow n}

\inference[\textsc{bool}]{(b\text{ is a boolean literal})}{(b, \rho) \Downarrow b}

\inference[\textsc{var}]{(\rho(x) = v)}{(x, \rho) \Downarrow v}

\inference[\textsc{op}]{(e_1, \rho) \Downarrow v_1 \and (e_2, \rho) \Downarrow v_2 \and (v_1 \oplus v_2 = v)}{(e_1\ \texttt{op}\ e_2, \rho) \Downarrow v} \text{ for each operator \texttt{op} and its meta-level equivalent }\oplus
\end{mathpar}

\begin{mathpar}

\inference[\textsc{asgn}]{(e, \rho) \Downarrow v}{(x\ \texttt{:=}\ e, \rho) \rightarrow (\texttt{skip}, \rho[x \mapsto v])}

\inference[\textsc{seq-struct}]{(c_1, \rho) \rightarrow (c_1', \rho')}{(c_1\texttt{;}\,c_2, \rho) \rightarrow (c_1'\texttt{;}\,c_2, \rho')}

\inference[\textsc{seq-comp}]{}{(\texttt{skip;}\,c_2, \rho) \rightarrow (c_2, \rho)}

\end{mathpar}
\section{Problems}

\begin{enumerate}
\item (6 points total) Consider the following program configuration: \[(\texttt{a := b + 2; c := 3 + 4},\; \{\texttt{b} = 5\})\]

\begin{enumerate}
\item (2 points) What is the top-level operation in this configuration's program? Put another way, which rule's conclusion would match the entire configuration?


\textit{In this configuration, we have two statements separated by a semicolon (;). According to the operational semantics rules provided, this corresponds to the "seq-struct" rule, which should allow for the execution of two statements in sequence. This rule's conclusion matches the entire configuration as it represents the execution of multiple statements in sequence within a block or program.
}

\vspace{.5in}


\item (4 points) Write a proof tree for the step taken by the configuration. The bottom of the tree should have the form $(\texttt{a := b + 2; c := 3 + 4},\; \{\texttt{b} = 1\}) \rightarrow ...$, with the $...$ filled in according to the rules you apply. You only need to write a proof tree for a single small step.


\textit{
As we are only looking for a single small step, we take the ASGN step and resolve the value of b
(a := b + 2; c := 3 + 4, {b = 1}) → (a := 1 + 2; c := 3 + 4)
}

\end{enumerate}
\newpage

\item (9 points total) Suppose we wanted to add a new \texttt{?} operation to our language of the form $x\ \texttt{?}\ y$, where $x$ and $y$ can be any program expression. The program $x\ \texttt{?}\ y$ is meant to return the value of $x$ unless that value is 0, in which case it returns the value of $y$ instead.

\begin{enumerate}
\item (2 points) Should $x\ \texttt{?}\ y$ be an expression or a command? Why?

\textit{
As "x ? y" can be used to evaluate values, it should be an expression.  "x ? y" will evaluate to a certain value given a condition (if x is 0).  We can use this program like so: a = x ? y and assign the value of a to the result.  As we compute a value and don't change the program state, an expression is most appropriate.}

\vspace{1in}

\item (5 points) Write one or more semantic rules for the \texttt{?} operation, in the appropriate style (big-step or small-step) based on your answer to the previous question. \newline


If E1 \Downarrow 0, and E2 \Downarrow v, then (E1 ? E2) \Downarrow v

If E1 \Downarrow v1 (v1 != 0), then (E1 ? E2) \Downarrow v1



\vspace{2.3in}

\item (2 points) Using your rules, write a proof tree showing that if the value of variable $\texttt{z}$ is currently 0 and the value of variable \texttt{a} is currently 5, then the value of \texttt{z ?\ a} is 5. \newline


Given:
The current state is: {z = 0, a = 5}

We want to prove: (z ? a) \Downarrow 5

\t Applying rule 1...

\t z \Downarrow 0 , a \Downarrow 5

\t Therefore, (z ? a) \Downarrow 5



\vspace{1.8in}


\item (for graduate students) Now, suppose we want $x\ \texttt{?}\ y$ to \textbf{change} the value of $x$ (which must be a variable) to the value of $y$ when $x$ is 0, instead of returning the value of $y$. When $x$ is not zero, $x\ \texttt{?}\ y$ should do nothing at all. Would this change your answer to 2a? How would your rule(s) in 2b change? \newline

Yes the answer to 2a would change.  If the behavior or "x ? y" is to change the value of variable "x" to the value of "y" when "x" is 0 (and it is implied that no-op occurs when x is not zero), then this would change "x ? y" from an expression to a command, as the state of the program is being changed. \newline

Here is how to semantic rules would change:

If E1 \Downarrow 0\  and\  x=VAR, and \ E2 \Downarrow v, \ then (x ? y) -> (skip, \rho[ 7 -> v])

(The value of x is change to v is E1 evaluates to 0.


If E1 \Downarrow v1 (v1 != 0), then (x ? y) -> skip

(No action taken if E1 is non-zero)


\end{enumerate}

\end{enumerate}

\end{document}